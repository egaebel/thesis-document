%
% PROJECT: <ETD> Electronic Thesis
%   TITLE: Looks Good to Me (LGTM): Authentication for Augmented Reality
%  AUTHOR: Ethan Gaebel
% SAVE AS: thesis-draft.tex
% 

\documentclass[12pt,dvips]{report}

\setlength{\textwidth}{6.5in}
\setlength{\textheight}{8.5in}
\setlength{\evensidemargin}{0in}
\setlength{\oddsidemargin}{0in}
\setlength{\topmargin}{0in}

\setlength{\parindent}{0pt}
\setlength{\parskip}{0.1in}

% Uncomment for double-spaced document.
% \renewcommand{\baselinestretch}{2}

% \usepackage{epsf}

\begin{document}

\thispagestyle{empty}
\pagenumbering{roman}
\begin{center}

% TITLE
{\Large 
Looks Good to Me (LGTM): 
Authentication for Augmented Reality Using Wireless Localization and Facial Recognition
}

\vfill

Ethan D. Gaebel

\vfill

Thesis submitted to the Faculty of the \\
Virginia Polytechnic Institute and State University \\
in partial fulfillment of the requirements for the degree of

\vfill

Master of Science \\
in \\
Computer Science and Applications

\vfill

Wenjing Lou, Committee Chair \\
Ing-Ray Chen \\
Guoqiang Yu 

\vfill

May 15, 2016 \\
Falls Church, Virginia

\vfill

Keywords: 
\\
Copyright 2016, Ethan D. Gaebel

\end{center}

\pagebreak

\thispagestyle{empty}
\begin{center}

{\large Looks Good to Me (LGTM): 
Authentication for Augmented Reality Using Wireless Localization and Facial Recognition}

\vfill

Ethan D. Gaebel

\vfill

(ABSTRACT)

\vfill

\end{center}

Security-based abstract and stuff!!!

\vfill

% GRANT INFORMATION
This work was supported in part by the National Science Foundation under Grants 
CNS-1443889, CNS-1405747, CNS-1446478, and CNS-1343222.

\pagebreak

% Dedication and Acknowledgments are both optional
% \chapter*{Dedication}
% \chapter*{Acknowledgments}

\tableofcontents
\pagebreak

\listoffigures
\pagebreak

\listoftables
\pagebreak

\pagenumbering{arabic}
\pagestyle{myheadings}

%%%%%%%%%%%%%%%%%
\chapter{Introduction}
\markright{Ethan D. Gaebel \hfill Chapter 1. Introduction \hfill}
% Augmented Reality
% ---- Convince readers it is the future
% ---- Discuss options for AR: headsets, hardware included, etc
% ---- Augmented reality NEEDS a device pairing mechanism
% Device Pairing
% ---- Difficulty of bootstrapping authentication
% ---- Man-in-the-middle (MOVE)
% ---- Evil twin (MOVE)
% ---- Difficulty of pairing over wireless channel 
% ---- Usability's importance to security in practice
% ---- Pin currently dominant method
% ---- Long history, briefly summarize to illustrate the breadth of techniques investigated
% ---- Mention centralized device pairing methods.... (this is kind of out of the purview of device pairing; a user account can be used to bootstrap?)
% Augmented reality needs device pairing
% ---- Sharing holograms, etc
% ---- Improved hardware and user interface provides avenue for BETTER device pairing (better = more usable --> more secure)
% -------- Must be more usable than a pin


% (TO BE ASSIGNED TO A CHAPTER)
% High level description of LGTM from the user perspective (high level even for the user view)
% ---- Press button, look at other person, etc
% ---- Compare EM waves & sound waves for localization & information carrying purposes
% ---- How does this look formally? (transition)


%%%%%%%%%%%%%%%%%
\chapter{Protocol Design}
% System Model
% ---- Protocol hardware requirements
% (Security) Objectives
% ----Bootstrapping authentication
% ----Selection problem
% ----Privacy objectives (No need for centralized power knowing communication/content patterns)
% Threat Model
% ---- Wireless attacker model
% Protocol, high level
% Protocol Figure
% Protocol, specific (Alice receives $n$ packets, etc...)
% Protocol Analysis
% ---- Describe each step of the protocol
% ---- Security
% -------- Attacker ability is based on the quality of the facial recognition & wireless localization
% -------- "Attack space"
%
% ---- Usability

%%%%%%%%%%%%%%%%%
\chapter{Implementation}
% General setup
% ---- AR headsets unavailable (expensive, not powerful enough yet, not flexible enough yet)
% ---- Necessary components from AR headsets for experimental purposes (Wireless, webcam, display)
% Wireless Localization Discussion
% ---- Discuss past localization techniques, BRIEFLY
% ---- Discuss lack of focus on point-to-point localization
% ---- Discuss new protocols that enable point-to-point localization in commodity hardware
% Facial Recognition Discussion
% ---- Very brief discussion of techniques, the area is well studied and my techniques are not new
% Specific Wireless localization and facial recognition methods
% ---- SpotFi
% ---- OpenCV
% -------- Fisherfaces, Haarcascade, etc
% Specific localization & facial recognition components
% ---- Intel 5300 firmware etc
% ---- Laptops with webcams and three antennas each
% System and Validation
% ---- Glue code, system structuring, specifics of protocol implementation
% ---- Explain each implementation decision, why and how, discuss alternatives
% ---- This section should mainly discuss why certain things were NOT done, why things ARE done 
% -------- belongs under "General Setup"

%%%%%%%%%%%%%%%%%
\chapter{Experiment}
% Experiments (Very formal)
% ----Security
% ----Performance
% ----SpotFi Accuracy
% ----Facial recognition accuracy
% Results (for each above)
% Discussion (for each above)

%%%%%%%%%%%%%%%%%
\chapter{Conclusion}
% List of points to hit on:
% ---- Oldness and difficulty of device pairing
% ---- Usability's impact on security
% ---- Ease of use of LGTM
% ---- Point-to-point communication providing privacy
% ---- Something optimistic and feel-good sounding about the future

%%%%%%%%%%%%%%%%%
%
% Include an EPS figure with this command:
%   \epsffile{filename.eps}
%

%%%%%%%%%%%%%%%%
%
% Do tables like this:

% \begin{table}
% \caption{The Graduate School wants captions above the tables.}
% \begin{center}
%  \begin{tabular}{ccc}
%  x & 1 & 2 \\ \hline
%  1 & 1 & 2 \\
%  2 & 2 & 4 \\ \hline
%  \end{tabular}
% \end{center}
% \end{table}

%%%%%%%%%%%%%%%%%%%%%%%%%%%%%%%%

% If you are using BibTeX, uncomment the following:
% \thebibliography
%
% Otherwise, uncomment the following:
% \chapter*{Bibliography}

% \appendix

% In LaTeX, each appendix is a "chapter"
% \chapter{Program Source}


\end{document}