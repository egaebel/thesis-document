%
% PROJECT: <ETD> Electronic Thesis
%   TITLE: Looks Good to Me (LGTM): Authentication for Augmented Reality
%  AUTHOR: Ethan Gaebel
% SAVE AS: thesis-draft.tex
% 

\documentclass[12pt,dvips]{report}

\setlength{\textwidth}{6.5in}
\setlength{\textheight}{8.5in}
\setlength{\evensidemargin}{0in}
\setlength{\oddsidemargin}{0in}
\setlength{\topmargin}{0in}

\setlength{\parindent}{0pt}
\setlength{\parskip}{0.1in}

% Uncomment for double-spaced document.
% \renewcommand{\baselinestretch}{2}

% \usepackage{epsf}

\begin{document}

\thispagestyle{empty}
\pagenumbering{roman}
\begin{center}

% TITLE
{\Large 
Looks Good to Me (LGTM): 
Authentication for Augmented Reality Using Wireless Localization and Facial Recognition
}

\vfill

Ethan D. Gaebel

\vfill

Thesis submitted to the Faculty of the \\
Virginia Polytechnic Institute and State University \\
in partial fulfillment of the requirements for the degree of

\vfill

Master of Science \\
in \\
Computer Science and Applications

\vfill

Wenjing Lou, Committee Chair \\
Ing-Ray Chen \\
Guoqiang Yu 

\vfill

May 15, 2016 \\
Falls Church, Virginia

\vfill

Keywords: 
\\
Copyright 2016, Ethan D. Gaebel

\end{center}

\pagebreak

\thispagestyle{empty}
\begin{center}

{\large Looks Good to Me (LGTM): 
Authentication for Augmented Reality Using Wireless Localization and Facial Recognition}

\vfill

Ethan D. Gaebel

\vfill

(ABSTRACT)

\vfill

\end{center}

Security-based abstract and stuff!!!

\vfill

% GRANT INFORMATION
This work was supported in part by the National Science Foundation under Grants 
CNS-1443889, CNS-1405747, CNS-1446478, and CNS-1343222.

\pagebreak

% Dedication and Acknowledgments are both optional
% \chapter*{Dedication}
% \chapter*{Acknowledgments}

\tableofcontents
\pagebreak

\listoffigures
\pagebreak

\listoftables
\pagebreak

\pagenumbering{arabic}
\pagestyle{myheadings}

%%%%%%%%%%%%%%%%%
\chapter{Introduction}
\markright{Ethan D. Gaebel \hfill Chapter 1. Introduction \hfill}
% Augmented Reality
% ---- Convince readers it is the future
% ---- Discuss options for AR: headsets, hardware included, etc
% ---- Augmented reality NEEDS a device pairing mechanism
% ---- Why pair point-to-point?
% -------- More efficient network resource usage
% --------- Problem is inherently local
% -------- Not reliant on infrastructure, more robust to patchy areas
% ------------ Cheaper if we consider the non-WiFi usage 
% ---------------- WAY cheaper if you further consider the content size!
% ---- Preserves privacy
% Augmented reality needs device pairing
% ---- Sharing holograms, etc
% Device Pairing
% ---- Difficulty of bootstrapping authentication
% ---- Man-in-the-middle (MOVE)
% ---- Evil twin (MOVE)
% ---- The Resurrecting Duckling (MOVE)
% ---- Long history, briefly summarize to illustrate the breadth of techniques investigated
% ---- Difficulty of pairing over wireless channel 
% ---- Usability's importance to security in practice
% ---- Pin currently dominant method
% ---- Mention centralized device pairing methods.... (this is kind of out of the purview of device pairing; a user account can be used to bootstrap?)
% Augmented reality can have BETTER device pairing
% ---- Improved hardware and user interface provides avenue for BETTER device pairing (better = more usable --> more secure)
% -------- Must be more usable than a pin
Augmented reality is one of the most exciting new technologies on the horizon, promising three dimensional interfaces that one or more users can directly interact with. Augmented reality is most commonly provided via a head mounted display with translucent lenses that serve as a display where three dimensional digital objects can be rendered on top of the real world. These head mounted displays are usually full stand-alone computers equipped with powerful processors and graphics processors, wireless communications chipsets, one or more high resolution cameras, and depth sensors. Head mounted displays require much of this sophisticated technology to provide fully interactive augmented reality which, at a minimum, requires: environment mapping, hand tracking, graphics processing, and wireless communication, since connectivity is a requirement for any device today. 
Users of augmented reality headsets have the same need that users of any other electronic device have today: the need to send content to others. But in the case of augmented reality this content will usually be far richer, and larger, than we've seen on other platforms, since users will be creating, viewing, and working with three dimensional objects, which inherently have more information associated with them than their equivalent two dimensional representations. It is also likely that users will want to share content face-to-face, providing 3D objects that both users can see and/or interact with; I'm sure most smart phone users have looked up a video or picture to show someone before. In fact, I think that we can expect this type of sharing to increase with augmented reality so that users are sharing app interfaces, sharing group photos in real-time, and passing secret three dimensional jokes back and forth all the time.
This content sharing between users in close proximity is an interesting problem. Typical content sharing schemes today take advantage of pre-existing infrastructure via cell towers, wireless access points, and the Internet. This makes very good sense considering the most common use-case for digitally sharing content is to share it with those who aren't present; you wouldn't send someone standing next to you a YouTube video. But with augmented reality you would send someone sitting next to you a YouTube video so that the two of you can watch it together through your respective headsets. Since this type of sharing will be so pronounced I think it deserves separate consideration from the general content sharing problem, which is why I think that point-to-point wireless communication should be used for such content sharing. By sharing content directly from person to person network resources are saved, money may be saved if charge-by-data-usage plans are being used for Internet access, person-to-person content sharing is robust in the face of local or global infrastructure failures, and content sharing privacy details, such as who is communicating with who, what is being communicated, and when the communication is occurring, are kept between the two users doing the sharing instead of being shared with whatever entity (or entities) is (are) providing the service. 
There are many options for point-to-point wireless communication but a full discussion and evaluation of these options is outside the purview of this thesis. But it is relevant to briefly mention some major enabling technologies so as to convince the reader of the feasibility of such a design, so I will briefly mention a few common technologies. WiFi Direct is perhaps the simplest since it is part of the any WiFi-capable device can take advantage of it with the right software. WiFi Direct also has direct API support in the popular Android operating systems. Bluetooth also includes a point-to-point protocol,  support in Android, iOS, and most modern computers. 
When working in a point-to-point wireless communications scenario authentication can be tricky. It will often be the case that two users have not used their devices to communicate before, so there will be no pre-existing security context (such as a pin, key, token, etc) between the two devices. 
Device pairing is the area dedicated to authenticating devices without prior security context, and there have been many schemes introduced to address this problem on devices with myriad hardware features and constraints. To give a quick idea of the breadth of the techniques proposed, some schemes have involved: screens with changing bar codes, infrared communication, blinking lights, communicating cryptographic keys using speakers and microphones, and numerical pin entry. All of these methods rely on communication over one or more out-of-band channels, which is a channel using human sensory capabilities to authenticate other communication channels which are imperceptible to humans, which is usually the wireless channel. To authenticate the human imperceptible channel some key is transmitted over the out-of-band channel which is then used for authentication. The simplest, and most common, example is that of a numerical pin. A pin may be permanently bound to a device without a user interface and the user enters the pin into another device with a user interface to perform authentication; this is the case with Bluetooth headsets.
These out-of-band channels explicitly require a human to participate, and it may be expected that the human actor has some impact on the security of the scheme as a result. This intuition has been validated in $[CITATIONS]$ where comparative user studies were conducted between several different device pairing techniques. In $[CITATION--Usability Analysis of Secure Device Pairing Methods]$ two rounds of usability tests were performed for two different methods: compare-and-confirm and select-and-confirm, where users had to compare a randomly generated pin presented on each device for equality and confirm if they matched and select a pin from a list on one device sent to it from the first device and confirm if they matched, respectively. Each method was measured by fatal error rate and total user error rate, where a fatal error is an error that results in a security breach and total user error is all the fatal errors combined with all errors that do not result in a security breach (only user annoyance). The two rounds of usability tests differed in the user interface the study participants used, while also being performed on different users. In the first round compare-and-confirm had a fatal error rate of $20\%$ and select-and-confirm had a fatal error rate of $12.5\%$, while compare-and-confirm had a $20\%$ total error rate and select-and-confirm had a $20\%$ total error rate. The researchers suspected that the user interface could be the problem and so improved the interfaces as well as lengthening the pins from 4 numbers to 6 numbers, the idea being that a longer pin will force the user to focus more on the task making them less likely to make a mistake. In the second round compare-and-confirm had a fatal error rate of $0\%$ and select-and-confirm had a fatal error rate of $5\%$, while compare-and-confirm had a $2.5\%$ total error rate and select-and-confirm had a $7.5\%$ total error rate. So we see that these results mirror our intuition regarding usability's effect on the security of device pairing schemes. Furthermore, making authentication easier for the user is a worthy goal in-of-itself. A hallmark of technology in general has been making certain things easier for the masses, so it should be with security technology. 
Augmented reality presents a new computing platform with many hardware features as a requirement that have not been previously seen on consumer devices. These hardware features present an opportunity for a better device pairing scheme that is both more secure and more usable. Think of how two people who have never met before authenticate one another while speaking. Sound waves are localized and paired with the face we see at the point of localization. Afterwards we know the sound of the person's voice and can recognize it without seeing them. Can we do something similar with the hardware available in an augmented reality headset? Yes. We can localize the wireless signal and overlay it on reality for the user to view, and use facial recognition to locate a face fitting a model transmitted to our device and overlay a box around the face recognized in the current frame, then the user can verify on the digital overlay over reality that the wireless signal is emanating from the person we would like to communicate with. If the user finds that the person our device recognizes is who they want to communicate with and that the wireless signal is indeed coming from them, the user can say to themselves ``looks good to me!'' and indicate that the other user has been authenticated. This is the basic view of how my system works that the users will interact with, but lets move on to the next section to formalize this protocol and everything that happens under the hood.

% (TO BE ASSIGNED TO A CHAPTER)
% High level description of LGTM from the user perspective (high level even for the user view)
% ---- Press button, look at other person, etc
% ---- Compare EM waves & sound waves for localization & information carrying purposes
% ---- How does this look formally? (transition)


%%%%%%%%%%%%%%%%%
\chapter{Protocol Design}
% System Model
% ---- Protocol hardware requirements
% (Security) Objectives
% ----Bootstrapping authentication
% ----Selection problem
% ----Privacy objectives (No need for centralized power knowing communication/content patterns)
% Threat Model
% ---- Wireless attacker model
% Protocol, high level
% Protocol Figure
% Protocol, specific (Alice receives $n$ packets, etc...)
% Protocol Analysis
% ---- Describe each step of the protocol
% ---- Security
% -------- Attacker ability is based on the quality of the facial recognition & wireless localization
% -------- "Attack space"
%
% ---- Usability

%%%%%%%%%%%%%%%%%
\chapter{Implementation}
% General setup
% ---- AR headsets unavailable (expensive, not powerful enough yet, not flexible enough yet)
% ---- Necessary components from AR headsets for experimental purposes (Wireless, webcam, display)
% Wireless Localization Discussion
% ---- Discuss past localization techniques, BRIEFLY
% ---- Discuss lack of focus on point-to-point localization
% ---- Discuss new protocols that enable point-to-point localization in commodity hardware
% Facial Recognition Discussion
% ---- Very brief discussion of techniques, the area is well studied and my techniques are not new
% Specific Wireless localization and facial recognition methods
% ---- SpotFi
% ---- OpenCV
% -------- Fisherfaces, Haarcascade, etc
% Specific localization & facial recognition components
% ---- Intel 5300 firmware etc
% ---- Laptops with webcams and three antennas each
% System and Validation
% ---- Glue code, system structuring, specifics of protocol implementation
% ---- Explain each implementation decision, why and how, discuss alternatives
% ---- This section should mainly discuss why certain things were NOT done, why things ARE done 
% -------- belongs under "General Setup"

%%%%%%%%%%%%%%%%%
\chapter{Experiment}
% Experiments (Very formal)
% ----Security
% ----Performance
% ----SpotFi Accuracy
% ----Facial recognition accuracy
% Results (for each above)
% Discussion (for each above)

%%%%%%%%%%%%%%%%%
\chapter{Conclusion}
% List of points to hit on:
% ---- Oldness and difficulty of device pairing
% ---- Usability's impact on security
% ---- Ease of use of LGTM
% ---- Point-to-point communication providing privacy
% ---- Something optimistic and feel-good sounding about the future

%%%%%%%%%%%%%%%%%
%
% Include an EPS figure with this command:
%   \epsffile{filename.eps}
%

%%%%%%%%%%%%%%%%
%
% Do tables like this:

% \begin{table}
% \caption{The Graduate School wants captions above the tables.}
% \begin{center}
%  \begin{tabular}{ccc}
%  x & 1 & 2 \\ \hline
%  1 & 1 & 2 \\
%  2 & 2 & 4 \\ \hline
%  \end{tabular}
% \end{center}
% \end{table}

%%%%%%%%%%%%%%%%%%%%%%%%%%%%%%%%

% If you are using BibTeX, uncomment the following:
% \thebibliography
%
% Otherwise, uncomment the following:
% \chapter*{Bibliography}

% \appendix

% In LaTeX, each appendix is a "chapter"
% \chapter{Program Source}


\end{document}